%%%%%%%%%%%%%%%%%%%%%%%%%%%%%%%%%%%%%%%%%%%%%%%%%%%%%%%%%%%%%%%%%%%%%%%%%%%%%%%%%%%%%%%%%%%%%%%%%%%%%%%%%%%%

{\LARGE \bf Dr. Séverin Lemaignan}{\Large, Associate Professor in Social Robotics and AI}

\vspace{1em}

\begin{tabular}{p{0.45\linewidth}p{0.45\linewidth}}
    \textbf{ORCID}:
    \href{http://orcid.org/0000-0002-3391-8876}{0000-0002-3391-8876} & \textbf{Date of birth}: 17 Jan 1983 (37 years old) \\
\textbf{Nationality}: French & \href{https://academia.skadge.org}{academia.skadge.org} -- \href{https://twitter.com/skadge}{twitter.com/skadge}
\end{tabular}

\vspace{1em}

\section{EDUCATION}

\begin{tabular}{p{0.15\linewidth}p{0.8\linewidth}}
    \bf 2008 -- 2012 & {\bf Joint German-French PhD in Cognitive Robotics}
    \newline LAAS-CNRS, France / Technical University of Munich, Germany
    \newline {\small Supervisors: Pr. Rachid Alami, CNRS; Pr. Michael Beetz,
    TUM} \\
    \bf 2004 -- 2005 &  {\bf MSc Artificial Intelligence for Learning
    Technologies}
    \newline University Paris V, France \\
    \bf 2002 -- 2002 & {\bf Joint German-French MSc of Engineering} \newline Karlsruhe
    Institute of Technology, Germany / ENSAM ParisTech, France \\
\end{tabular}

\section{EMPLOYMENT}

\begin{tabular}{p{0.15\linewidth}p{0.8\linewidth}}
    \bf 2019 -- & {\bf Associate Professor in Social Robotics and Artificial
    Intelligence}
    \newline Bristol Robotics Laboratory, University of the West of England,
    United Kingdom 
    \newline \small Supervision of the Human-Robot Interaction research group; Supervision of the Driverless Vehicle research group.
Directly managing 20+ students and early career researchers. \\

    \bf 2018 -- 2019 & {\bf Senior Research Fellow in Robotics and AI} \newline Bristol Robotics Laboratory, University of the West of England, United Kingdom \\
    \bf 2017 -- 2018 & {\bf Lecturer in Robotics} \newline Plymouth University, Plymouth, United Kingdom \\
    \bf 2015 -- 2017 & {\bf EU Marie Skłodowska-Curie Post-doctoral fellow}
    \newline Plymouth University, Plymouth, United Kingdom \newline \small
    Development and Implementation of a Theory of Mind for robots \\
    \bf 2013 -- 2015 & {\bf Post-doctoral researcher} \newline CHILI, EPFL,
    Lausanne, Switzerland \newline \small Interaction with Robots in Learning
    Environments – Supervision of the robotic group \\
    \bf 2012 -- 2013 & {\bf Post-doctoral researcher} \newline LAAS-CNRS, Toulouse,
    France \newline \small Spatial and Temporal Reasoning for Cognitive Robotic
    Architectures\\
    \bf 2006 -- 2007 & {\bf Research Engineer} \newline INRIA, Paris, France
    \newline \small Development of semantic-aware control architectures for
    autonomous vehicles \\
\end{tabular}


\section{FELLOWSHIPS AND AWARDS}

\begin{tabular}{p{0.15\linewidth}p{0.8\linewidth}}
    \bf 2019 & UWE Vice Chancellor Accelerator Fellowship \\
    \bf 2015 -- 2017 & {\bf EU Marie Skłodowska-Curie Individual Fellowship}
    \newline Theory of Mind and social robotics, Plymouth University, UK \\
    \bf HRI'2017  & Best Paper award\\
    \bf HRI'2016  & Best Paper award\\
    \bf AAAI'2015  & Best Video award in Artificial Intelligence\\
    \bf HRI'2014  & Best Late Breaking Report award\\
    \bf 2012         & {\bf Best PhD in Robotics 2012} award, CNRS, France \\
    \bf 2012         & PhD with {\bf High Distinction} (“Summa Cum Laude”), TU Munich\\
    \bf Ro-Man'2010  & Best paper award\\
\end{tabular}


\section{SUPERVISION OF GRADUATE STUDENTS AND POSTDOCTORAL FELLOWS}

\begin{tabular}{p{0.15\linewidth}p{0.8\linewidth}}
    \bf 2018 -- 2019 & \textbf{2 post-docs}, \textbf{5 PhDs}, \textbf{4 MSc students}, Bristol Robotics Lab, UWE, UK \\
    \bf 2015 -- 2018 & \textbf{3 PhDs}, Plymouth University, UK \\
    \bf 2013 -- 2015 & \textbf{5 PhDs}, \textbf{5 MSc students}, EPFL, Switzerland \\
    \bf 2012 -- 2013 & \textbf{2 MSc students}, LAAS-CNRS, France \\
\end{tabular}


\section{TEACHING ACTIVITIES}

\begin{tabular}{p{0.15\linewidth}p{0.8\linewidth}}
    \bf 2019 --  & \textbf{Associate Professor} (postgraduate; HRI), Bristol Robotics Lab, UWE, UK \\
    \bf 2018 -- 2019 & \textbf{Senior Lecturer} (postgraduate; HRI), Bristol Robotics Lab, UWE, UK \\
    \bf 2015 -- 2018 & \textbf{Lecturer} (undergraduate \& postgraduate; robotics
    fundamentals, software engineering, human-robot interaction), Plymouth University, UK \\
    \bf 2013 -- 2015 & \textbf{Teaching Assistant} (undergraduate; Visual Computing), EPFL, Switzerland \\
    \bf 2008 -- 2012 & \textbf{Teaching Assistant} (undergraduate; programming, databases, ontologies), INSA Toulouse, France \\
\end{tabular}

\section{ORGANISATION OF SCIENTIFIC MEETINGS}

\begin{tabular}{p{0.05\linewidth}p{0.9\linewidth}}
    \bf 2020 & \textbf{ACM/IEEE Human-Robot Interaction conference}, 700+ participants, local chair, Cambridge, UK \\
    \bf 2017 & \textbf{ACM/IEEE Human-Robot Interaction conference}, 400+
    participants, alt.HRI chair, Vienna, AT \\
    \bf 2016 & \textbf{2nd Intl. workshop on Cognitive Architecture for Social HRI}, 45 participants, programme chair, Christchurch, NZ \\
    \bf 2014 & \textbf{Intl. workshop on Simulation for HRI}, 35 participants, programme chair, Bielefeld, DE \\
    \bf 2012 & \textbf{Intl. workshop on MORSE and its applications}, 30 participants, programme chair, Toulouse, FR \\
    \bf 2009 & \textbf{Cognitive Sciences’ Young Researchers Conference}, 150 participants, steering committee, Toulouse, FR \\
\end{tabular}

\section{INSTITUTIONAL RESPONSIBILITIES}

\begin{tabular}{p{0.15\linewidth}p{0.8\linewidth}}
    \bf 2019 -- & Associate Professor, Faculty of Technology and Environment, UWE, UK \\
    \bf 2019 -- & Head of the Outreach cluster, Faculty of Technology and Environment, UWE, UK \\
    \bf 2019 & PhD defense committee, University of Bielefeld, DE \\
    \bf 2019 & PhD defense committee, University of Örebro, SE \\
    \bf 2018 -- & HRI module co-lead, MSc level, University of the West of England, UK  \\
    \bf 2017 -- 2018 & Module leader, Robotics fundamentals (undergraduate level), University of Plymouth, UK \\
\end{tabular}

\section{EDITORIAL ACTIVITIES}

\begin{tabular}{p{0.15\linewidth}p{0.8\linewidth}}
    \bf 2018 -- & Editorial board of \emph{Frontiers in AI and Robotics} \\
    \bf 2015 --  & Member of the IEEE/ACM HRI Programme Committee \\
    \bf 2019 --  & Member of the Robotics, Science and System (RSS) Programme Committee  \\
    \bf 2017 -- 2019 & Member of the IEEE IROS Programme Committee  \\
    \bf 2017 -- 2018 & Member of the IJCAI Programme Committee  \\
    \bf 2017 -- 2018 & Member of the HAI Programme Committee  \\
\end{tabular}

\vspace{2em}

\section{PERSONAL STATEMENT}\label{early-achievements-track-record}

Since I completed my joint PhD in Cognitive Robotics from the CNRS/LAAS (France) and the
Technical University of Munich (Germany), for which I received the \emph{Best
PhD in Robotics 2012} award from French CNRS and the prized \emph{Cumma Summa
Laude} distinction in Germany, I have emerged as a leading authority in
HRI.

Soon after my PhD, I created and successfully led for 2 years the HRI group
within the AI for Learning CHILI Lab at EPFL (Switzerland), supervising in total
10 students, and establishing in a short timeframe CHILI as an internationally
recognised centre in robotics for education. While my original training was in
\textbf{symbolic cognition \& AI for autonomous robotics}, my postdoctoral stay
at the highly cross-disciplinary CHILI Lab gave me the opportunity to become an
expert in \textbf{experimental sciences, socio-psychology and education
sciences}.

I was then awarded an EU \textbf{H2020 Marie Sklodovska-Curie Individual
Fellowship} and I engaged in basic research on artificial cognition: over 2
years, I explored the underpinnings of artificial social cognition. I
\textbf{contributed significantly to the framing of the emerging field of
data-driven HRI}, also releasing of the PInSoRo open dataset
(\href{https://doi.org/10.5281/zenodo.1043507}{10.5281/zenodo.1043507}), a
\textbf{one-in-a-kind dataset of natural child-child and child-robot social
interactions}.

%Quickly after the fellowship, I was offered a permanent position
%as lecturer in robotics at Plymouth University, followed as a permanent Senior
%Research Fellow position at the Bristol Robotics Lab.
My current role as a permanent \textbf{Associate Professor in Social Robotics
and AI} at the Bristol Robotics Laboratory (BRL, largest co-located robotic lab
in the UK) recognise my leadership. I am \textbf{in charge of defining and
implementing the lab's research strategy in human-robot interactions}. I created
the Embedded Cognition for Human-Robot Interactions (ECHOS) research group, that
I now co-lead, supervising 15+ PhDs and post-docs. I also supervise the BRL's
Connected Autonomous Vehicles research group (5 students and post-docs).
Specifically, the ECHOS group covers most aspects of situated AI for human-robot
interaction, \textbf{my role includes strategic planning of the group
activities, scientific guidance, recruitment of staff and prospective students,
and grant applications}.

My field of expertise covers \textbf{the socio-cognitive aspects of
human-robot interaction, both from the perspective of the human cognition and
the design and implementation of cognitive architectures for robots}. I have
also focused a significant portion of my \textbf{experimental work on
child-robot interactions in real-world educative settings}, exploring how robots
can support teachers and therapists to develop engaging novel
learning paradigms.

This expertise is recognised internationally: I have a substantial track record
of academic outputs. Since 2008, I have authored or co-authored \textbf{75+ peer-reviewed
publications} in international journals and conferences, leading to \textbf{2500+
citations}, h-index of 25, i10-index of 40 (source: Google Scholar).

I have established strong \textbf{peer recognition} in the field of human-robot interaction
and cognitive robotics. For instance:

\begin{itemize}[noitemsep,topsep=0pt,parsep=0pt,partopsep=0pt]
    \item invited to \textbf{high-profile editorial roles}: Programme Committee member of the HRI
conference since 2015; editor of Frontiers In Robotics and AI journal; editor or
Programme Committee member of several leading conferences in AI and Robotics
        (RSS, IROS, IJCAI, HAI, AAMAS);
    \item invited member of the UK EPSRC Associate Peer Review College;
        invited reviewer for the French, Dutch, Israeli research agencies;
    \item numerous \textbf{invited talks} at national and international symposiums and
        events (9 invited talks since Jan. 2018, including \textbf{keynotes} at the UK Robotics
and Autonomous Systems 2019 conference, and at the 2018 AAAI Fall Symposium);
    \item \textbf{local chair for the high-profile, international HRI2020
        conference} (700+ delegates);
    \item regularly invited to PhD defense committees (most recently at
        LAAS, France, Uni Bielefeld, Germany, and Uni Örebro, Sweden).
\end{itemize}


\vspace{1em} 

I \textbf{actively engage with policy makers, at national and European
level}: for instance, over the past 2 years, I have been directly interacting
(through participating to panels, visits and one-to-one discussions) with the EU
Research Executive Agency (MSCA AI Cluster 2019); the UK minister for Business,
Energy and Industrial Strategy Greg Clark; the UK minister for Universities,
Science, Research and Innovation Chris Skidmore; the chair of the West of
England authority Tim Bowles; the UK Research \& Innovation Portfolio
manager for Robotics Clara Morri.

I have a \textbf{strong track record of tech transfer}, through patenting (US patent
US20190016213A1) and involvement in national and EU-level projects focused on
tech-transfer (InnovateUK ROBOPILOT, CAPRI, CAVForth; EU Terrinet, SABRE).

Finally, I actively engage in \textbf{research communication}: my past research has been
covered several times by mainstream international media, including press
releases by Reuters, Press Association; TV coverage by the BBC, Sky News; radio
interviews and broadcast. My academic website (\url{academia.skadge.org})
showcases this media coverage. I also maintain an active, science-focused,
presence on the social media (Twitter handle: @skadge).




%\section{MAJOR COLLABORATIONS}

%Name of collaborators, Topic, Name of Faculty/ Department/Centre, Name of University/ Institution/ Country

%%%%%%%%%%%%%%%%%%%%%%%%%%%%%%%%%%%%%%%%%%%%%%%%%%%%%%%%%%%%%%%%%%%%%%%%%%%%%%%%%%%%%%%%%%%%%%%%%%%%%%%%%%%%
%\section{}

%%%%%%%%%%%%%%%%%%%%%%%%%%%%%%%%%%%%%%%%%%%%%%%%%%%%%%%%%%%%%%%%%%%%%%%%%%%%%%%%%%%%%%%%%%%%%%%%%%%%%%%%%%%%
%\section{}

%%%%%%%%%%%%%%%%%%%%%%%%%%%%%%%%%%%%%%%%%%%%%%%%%%%%%%%%%%%%%%%%%%%%%%%%%%%%%%%%%%%%%%%%%%%%%%%%%%%%%%%%%%%%
%\section{}

