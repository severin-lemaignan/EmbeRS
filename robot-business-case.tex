\chapter{Equipment Business Case}

\epsrc{This section must follow the structure presented here:
\url{https://epsrc.ukri.org/research/facilities/equipment/process/researchgrants/}}

\textbf{PI name}: Séverin Lemaignan

\textbf{Host Institution}: University of the West of England

\textbf{Item}: Two mobile social robots with intrisic safety for close interaction with humans. The HALODI EVEr3 is sucha  platform.

\textbf{Vendor}: HALODI (the manufacturer of the robot)

\textbf{Description}: Briefly describe the item of equipment and its primary functions. Please explain how the specifications of the instrument make it different from other similar equipment available for use.

The key features that are necessary for the project are:

\begin{enumerate}
    \item an intrinsically safe platform, suitable for operation in close contact with humans, including children
    \item advanced social features, including an expressive face with mobile eyes
    \item a agile mobile platform, suitable for (potentially crowded) human environments
\end{enumerate}


Looking specifically at human-sized mobile manipulators with advanced social
features, the choice of robotic platforms is in effect limited.
Table~\ref{robot-comparison} compares the two leading mobile social robots
available on the market today (PAL TiaGo and SoftBank Pepper), along with the
IIT R1 platform developed by the Italian Institute of Technology and the HALODI
EVEr3. The Fetch Mobile Manipulator has been omitted, as it is functionally
similar to TiaGo.

Note that, even though IIT would offer early access to the its platform,
the IIT R1 robot is not yet officially available for purchase. As such, the
platform can be considered to be a prototype and might face teething issues.

\begin{table}[h!]
    \caption{Comparison of the HALODI EVEr3 with IIT R1, PAL TiaGo and Softbank
    Pepper. EVEr3 has been chosen for \project  for being the only mobile dual manipulator with good navigation and advanced social interaction
    capabilities.}
    \resizebox{\linewidth}{!}{
    \begin{tabular}{@{}p{3cm}p{5cm}p{5cm}p{5cm}p{5cm}@{}}
\toprule
                                           & \textbf{HALODI EVEr3}                                                                                                    & PAL TiaGo                                                & Softbank Pepper                                                           & IIT R1                                                                                                                              \\ \midrule
\textbf{Social features}                   & Expressive face;                                                                                                         & Poor (non-expressive head)                               & Expressive, yet fixed, face; limited gaze; approachable                   & Expressive face~\cite{lehman2016head}; artificial skin for touch-based interactions; approachable                                            \\
\textbf{Perception}                        & Good (SotA RGB-D; 7-mic array)                                                                                           & Medium (RGB-D camera; laser scanner; no microphone)      & Medium (RGB-D; simple mic array; poor laser scanner)                      & Good (RGB-D; simple mic array; laser scanner)                                                                                                \\
\textbf{Navigation}                        & Good (high agility due to Segway-like self-balancing; lack of laser scanner might impair floor-level obstacle detection) & Good (however, limited agility due to large footprint)   & Poor (weak localisation capabilities)                                     & Good (high agility due to Segway-like self-balancing)                                                                                        \\
\textbf{Safety}                            & Good (smaller footprint; safe arms with torque control; dynamic stability)                                               & Medium (heavy robot; large footprint; non-compliant arm) & Medium (smaller footprint; safe arms; limited stability)                  & Good (smaller footprint; safe arms; dynamic stability)                                                                                       \\
\textbf{Manipulation capabilities}         & Good (dexterous hand; dual arm; 6kg payload)                                                                             & Medium (non-anthropomorphic gripper; single arm)         & Limited (poor gripper with low payload; dual arm)                         & Good (anthropomorphic gripper; pressure sensors; dual arm; 1.5kg payload)                                                                    \\
\textbf{Suitability for care environments} & Good (designed for such domain, smaller footprint, easy to clean)                                                        & Poor (relatively large, difficult to clear)              & Good (smaller footprint, easy to clean)                                   & Good (smaller footprint, easy to clean)                                                                                                      \\ \bottomrule
\end{tabular}
}
    \label{robot-comparison}
\end{table}

\textbf{Cost}: 

\epsrc{Please set out the expected cost of the item(s) of equipment in pounds sterling
(inclusive of VAT) and list any associated maintenance or support costs.
Less expensive items of equipment that are intrinsically associated with the
equipment may also be requested. Funding for maintenance of the equipment or for
implementing or managing any sharing mechanisms can also be requested.
In order to assist EPSRC's financial planning, please state clearly the
anticipated funds that would be requested from EPSRC (in other words cost after
the deduction of any contributions) making a clear distinction between cost of
equipment item(s) and the cost of any associated resources. Please describe the
timescales associated with procurement of the equipment and when you anticipate
you will spend any capital provision made.}

\begin{table}[]
\begin{tabular}{@{}llrr@{}}
\toprule
\textbf{Item}                      & \textbf{Quantity}     & \multicolumn{1}{l}{\textbf{Unit Cost}} & \multicolumn{1}{l}{\textbf{Total cost}} \\ \midrule
EVEr3 robot                        & \multicolumn{1}{r}{1} & \textit{£124,530}                      & £124,530                                \\
EVEr3-HA1 hands                    & \multicolumn{1}{r}{2} & \textit{£8,510}                        & £17,020                                 \\
\textbf{Total (excl. maintenance)} &                       &                                        & \textbf{£141,550}                       \\
Maintenance                        & (per year)            & £11,520                                &                                         \\ \bottomrule
\end{tabular}
\end{table}

The maintenance costs include annual maintenance and on-site support and service.

The full cost of the robot and hands (£141,550) is requested from the EPSRC at
the start of the project. The robot's production lead time is about 4 months.
Taking into account the procurement time, we anticipate that the maintenance
costs will only have to be paid for a total duration of 4.5 years (paid on a
yearly basis).


\textbf{Usage}:

\epsrc{Indicate the proportion of equipment time that will be available for use by the
group managing the equipment, other groups at the same institution, and
researchers at other institutions. Indicate how additional users of the
equipment will be identified and how all users will be prioritised. Information
should be provided on the anticipated demand, indicating the likely main users
and where they will be based. You should describe how you plan to interact
and/or collaborate with key groups or shared facilities in your research area.
Note that it is acceptable for the equipment to be used entirely by one research
group although this would need to be very carefully justified. EPSRC are looking
to maximise the usage of equipment for high quality research, not necessarily
share it as widely as possible.}

Due to the nature of the project (in-the-wild deployement of the robot over
extended periods of time), the robot will be exclusively used by the project.

At the end of the project, we anticipate that the robot will naturally
fold into the numerous human-robot and assistive robotics activities taking
place at the Bristol Robotics Lab. As the Halodi EVEr3 robot belongs to the
latest generation of social robots (alongside the IIT R1, for example), it will
also contribute to the replacement of the ageing Pepper and TiaGo robots
currently extensively used in the laboratory.


\textbf{Support}: 

\epsrc{Please indicate how the item of equipment will be supported and maintained for
the duration of any current or proposed research funding including any costs
that would be recouped through charging.}

In addition to the maintenance contract that will provide operational support
over the course of the project, the Bristol Robotics Lab has a large technical
team with long and extensive experience in supporting a wide range of robots.
Dedicated technician time (3 months) is costed in the project, for specific
technical developments or liaising with the Halodi company.

\textbf{Strategic Case}:

\epsrc{Please describe the research enabled by the equipment and/or the value added to
existing research programmes. Indicate which of the EPSRC's strategic priorities
are met by the research enabled by this equipment.  Describe how these
priorities are met.  Explain how the purchase of this item of equipment will
compliment or enhance regional and/or national research capability. If an exists
for the type of equipment requested, please explain how this item relates to the
roadmap.}


\textbf{Ensuring Maximum Value}:

Explain how the requested equipment will fit with other items of equipment,
infrastructure and people support already at your university. Please indicate
how the requested item(s) fits the strategy of your department and institution.

\textbf{Contribution from Other Sources}:

\epsrc{Please describe here what contributions to the
cost, operation or maintenance of the item of equipment will be found from other
sources.}



\textbf{Alternatives}:

\epsrc{In the case that the proposal for equipment is not supported, describe the
alternative options for using existing equipment of different specification or
at other locations.}

Current-generation social robots (like PAL TiaGo or Softbank Pepper, both
available at the host institution) might be
explored as alternatives. The PI (and the laboratory) have very extensive
experience with both these robots.

Both these robots are generally suitable for deployements in real-world human
environments (even though PAL TiaGo would require specific safety measures as
its arm is not compliant, and therefore potentially dangerous).

However, they are not considered agile social platform: Softbank Pepper is known
to have poor navigation performances; they feature a dated set of sensors and
limited on-board processing capabilitiesN; nither TiaGo nor Pepper have
non-verbal social features that are powerful enough to deliver the WizUs
project. Critically, they both lack the abilities to show facial expressions or
simulated gazing behaviours. Because the EVEr3 robot features a programmable
display in place of the head, we will have full freedom to create complex
non-verbal facial expressions.

Besides, EVEr3 has been designed from the ground-up to be used in care
environments (in particular, hospital), and is made of materials that can easily
be cleaned up/disinfected. This is of critical importance for the deployment in
the hospital, or more generally in a post-COVID environment.



