\chapter{Equipment Business Case}

\epsrc{This section must follow the structure presented here:
\url{https://epsrc.ukri.org/research/facilities/equipment/process/researchgrants/}}

\textbf{PI name}: Séverin Lemaignan

\textbf{Host Institution}: University of the West of England

\textbf{Item}: Two mobile social robots with intrisic safety for close interaction with humans. The HALODI EVEr3 is sucha  platform.

\textbf{Vendor}: HALODI (the manufacturer of the robot)

\textbf{Description}: Briefly describe the item of equipment and its primary functions. Please explain how the specifications of the instrument make it different from other similar equipment available for use.

The key features that are necessary for the project are:

\begin{enumerate}
    \item an intrinsically safe platform, suitable for operation in close contact with humans, including children
    \item advanced social features, including an expressive face with mobile eyes
    \item a agile mobile platform, suitable for (potentially crowded) human environments
\end{enumerate}


Looking specifically at human-sized mobile manipulators with advanced social
features, the choice of robotic platforms is in effect limited.
Table~\ref{robot-comparison} compares the two leading mobile social robots
available on the market today (PAL TiaGo and SoftBank Pepper), along with the
IIT R1 platform developed by the Italian Institute of Technology and the HALODI
EVEr3. The Fetch Mobile Manipulator has been omitted, as it is functionally
similar to TiaGo.

Note that, even though IIT would offer early access to the its platform,
the IIT R1 robot is not yet officially available for purchase. As such, the
platform can be considered to be a prototype and might face teething issues.

\begin{table}[h!]
    \caption{Comparison of the HALODI EVEr3 with IIT R1, PAL TiaGo and Softbank
    Pepper. EVEr3 has been chosen for \project  for being the only mobile dual manipulator with good navigation and advanced social interaction
    capabilities.}
    \resizebox{\linewidth}{!}{
    \begin{tabular}{@{}p{3cm}p{5cm}p{5cm}p{5cm}p{5cm}@{}}
\toprule
                                           & \textbf{HALODI EVEr3}                                                                                                    & PAL TiaGo                                                & Softbank Pepper                                                           & IIT R1                                                                                                                              \\ \midrule
\textbf{Social features}                   & Expressive face;                                                                                                         & Poor (non-expressive head)                               & Expressive, yet fixed, face; limited gaze; approachable                   & Expressive face~\cite{lehman2016head}; artificial skin for touch-based interactions; approachable                                            \\
\textbf{Perception}                        & Good (SotA RGB-D; 7-mic array)                                                                                           & Medium (RGB-D camera; laser scanner; no microphone)      & Medium (RGB-D; simple mic array; poor laser scanner)                      & Good (RGB-D; simple mic array; laser scanner)                                                                                                \\
\textbf{Navigation}                        & Good (high agility due to Segway-like self-balancing; lack of laser scanner might impair floor-level obstacle detection) & Good (however, limited agility due to large footprint)   & Poor (weak localisation capabilities)                                     & Good (high agility due to Segway-like self-balancing)                                                                                        \\
\textbf{Safety}                            & Good (smaller footprint; safe arms with torque control; dynamic stability)                                               & Medium (heavy robot; large footprint; non-compliant arm) & Medium (smaller footprint; safe arms; limited stability)                  & Good (smaller footprint; safe arms; dynamic stability)                                                                                       \\
\textbf{Manipulation capabilities}         & Good (dexterous hand; dual arm; 6kg payload)                                                                             & Medium (non-anthropomorphic gripper; single arm)         & Limited (poor gripper with low payload; dual arm)                         & Good (anthropomorphic gripper; pressure sensors; dual arm; 1.5kg payload)                                                                    \\
\textbf{Suitability for care environments} & Good (designed for such domain, smaller footprint, easy to clean)                                                        & Poor (relatively large, difficult to clear)              & Good (smaller footprint, easy to clean)                                   & Good (smaller footprint, easy to clean)                                                                                                      \\ \bottomrule
\end{tabular}
}
    \label{robot-comparison}
\end{table}







\textbf{Cost}: 


Please set out the expected cost of the item(s) of equipment in pounds sterling (inclusive of VAT) and list any associated maintenance or support costs.

Less expensive items of equipment that are intrinsically associated with the equipment may also be requested. Funding for maintenance of the equipment or for implementing or managing any sharing mechanisms can also be requested.

In order to assist EPSRC's financial planning, please state clearly the anticipated funds that would be requested from EPSRC (in other words cost after the deduction of any contributions) making a clear distinction between cost of equipment item(s) and the cost of any associated resources. Please describe the timescales associated with procurement of the equipment and when you anticipate you will spend any capital provision made.
Usage: Indicate the proportion of equipment time that will be available for use by the group managing the equipment, other groups at the same institution, and researchers at other institutions. Indicate how additional users of the equipment will be identified and how all users will be prioritised. Information should be provided on the anticipated demand, indicating the likely main users and where they will be based. You should describe how you plan to interact and/or collaborate with key groups or shared facilities in your research area.

Note that it is acceptable for the equipment to be used entirely by one research group although this would need to be very carefully justified. EPSRC are looking to maximise the usage of equipment for high quality research, not necessarily share it as widely as possible.
Support: Please indicate how the item of equipment will be supported and maintained for the duration of any current or proposed research funding including any costs that would be recouped through charging.
Strategic Case: Please describe the research enabled by the equipment and/or the value added to existing research programmes. Indicate which of the EPSRC's strategic priorities are met by the research enabled by this equipment. Describe how these priorities are met. Explain how the purchase of this item of equipment will compliment or enhance regional and/or national research capability. If an
exists for the type of equipment requested, please explain how this item relates to the roadmap.
Ensuring Maximum Value: Explain how the requested equipment will fit with other items of equipment, infrastructure and people support already at your university. Please indicate how the requested item(s) fits the strategy of your department and institution.
Contribution from Other Sources: Please describe here what contributions to the cost, operation or maintenance of the item of equipment will be found from other sources.
Alternatives: In the case that the proposal for equipment is not supported, describe the alternative options for using existing equipment of different specification or at other locations.
