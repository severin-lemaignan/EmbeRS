\chapter{Justification of resources}


\section{Research team and PI commitment}

Table~\ref{time-allocation-team} provides an overview of the time allocation per
members of the team, over the course of the project.

\begin{table}[h!]
    \centering
\begin{tabular}{@{}lccccccr@{}}
\toprule
\textit{\textbf{}}              & \textbf{Y1} & \textbf{Y2} & \textbf{Y3} & \textbf{Y4} & \textbf{Y5} &  & \textbf{Total months} \\ \midrule
\textit{Séverin Lemaignan (PI)} & 0.6         & 0.6         & 0.6         & 0.6    & 0.6         &  & 36                    \\ \midrule
\textit{PDRA 1 (WP1)}       & 1           & 1           & 1           &             &             &  & 36                    \\
\textit{PDRA 2 (WP2)}       & 1           & 1           & 1           & 1           &             &  & 48                    \\
\textit{PDRA 3 (WP3)}       &             & 1           & 1           & 1           & 1           &  & 48                    \\
\textit{PDRA 4 (WP4, WP5)}  & 1           & 1           & 1           & 1           & 1           &  & 60                    \\
\textit{PhD 1 (WP4, WP5)}       &             & 1           & 1           & 1           & 0.5         &  & 42                    \\ \bottomrule
\end{tabular}
    \caption{Full-time equivalent for the research team members}
    \label{time-allocation-team}
\end{table}




\subsection{Team}

PI Séverin Lemaignan will dedicate 60\% (3 days/week) of his time to the project. This time will cover 
significant research time (about 2 days/week) as well as the supervision of the team and management 
of the project (1 day/week).

The rest of his time will be dedicate to other academic commitments within the Bristol Robotics Lab 
(including the on-going supervision of his other PhD students, supervision of MSc students, the 
supervision of the Human-Robot Interaction research group at BRL, lab-wide strategic engagement), 
as well as a small proportion of Master-level teaching in Human-Robot Interaction (about 5 days/term).

Each of the project work packages will have one lead researcher (post-doctoral
research assistant, PDRA); the duration of each
 of the PDRAs' contracts roughly matches the duration of the corresponding work packages.

-   WP1: I will appoint a PDRA (PD1) with a background in sociology of
    technology and science facilitation; the researcher will work for three year to
    frame the \emph{robot-supported human-human interactions} paradigm, and lead the
    field work at the WeTheCurious science centre (to this end, the centre has
    committed to provide in-kind training in science communication to the
    researcher, enabling her/him to engage directly with the public);

-   WP2: WP2 will be led by a PDRA (PD2) with a background in social signal processing and/or machine 
    learning; the researcher will be appointed for 4 years; extensive collaboration with WP1's PDRA
    is expected to frame the social dynamics fostered by the robot;

-   WP3: one PDRA (PD3, background in learning from demonstration and machine learning) will be in charge 
    of developing the novel continuous robot behaviour generation method, and will be appointed for 4 years, 
    starting on the second year;

-   WP4: WP4 (the cognitive architecture) lays at the core of the project; the WP4 leader will be a senior 
    PDRA in cognitive robotics (PD4), appointed for the whole 5 years to ensure continuity on this critical part; 
    she/he will be responsible for the integration of the outputs of the other work packages; the same PDRA will
    also oversee (with the PI) the experimental work taking place in WP5.

The cost for the WP4 PhD student (PHD1) is \textbf{not requested}, as the host
laboratory is part of the UK FARSCOPE Centre for Doctoral training, which will
fund the student directly.

In addition, a small amount of budget is allocated to senior staff Dr. Dave
Meckin (3 months FTE, support soundscape design, WP3.3) and Dr. Nigel Newbutt (4
months FTE, support the work in the SEN schools, WP5.1). I also have 3 month FTE
of technician time allocated over the duration of the project to support
specific technical developments on the robot.

\subsection{Research equipment}

I will purchase one Halodi EVEr3 robot (total £141,550) for the \project
project. The EVEr3 robot is a recently developed service robot from the Finish
Halodi company. The Equipment Business Case provides an extensive justification
for this robot.

\subsection{Travels}

Travels and conference fees have been costed on the basis of one international
conference per year and per person.

In addition, the budget include the costs of the four 2-days ethics workshops,
that the three members of the ethics Advisory Board will be invited to join.

\subsection{Subcontracting}

The subcontracting amount covers:
- the specific content creation and public communication costs, required to 
integrate the robot in the Bristol science centre WeTheCurious.
- work with the choreographer from the RustySquid
(http://www.rustysquid.org.uk/) company.

\subsection{Open access}

In line with the European requirements, all journal publications will made
available under an Open Access license.  On the basis of an average of 2 journal
publications per annum, and an average processing fee of €1,200 per article, we
request €12,000 to support Open Access costs. Note that conference publications
do not always offer immediate open-access policies.

\subsection{Other costs}

The €5000 cost in section A.3 correspond to the project auditing.

Consumables include cloud computing resources, organisation of 
the ethics advisory board workshops, participant compensations.

\subsection{Existing resources available to the researcher}

The fellowship will take place at the Bristol Robotics Laboratory (BRL). The BRL
is the largest co-located and most comprehensive advanced robotics research
establishment in the UK. It is a joint venture between the University of the
West of England and the University of Bristol.
BRL has an international reputation as a leading research centre in advanced
robotics research and has over 250 researchers working on a broad portfolio of
topics, including HRI, collective robotics, neuro-inspired control,
haptics, control systems, assistive robotics, soft robotics and biomedical
systems. This multidisciplinary environment will directly benefit the
project. BRL has many collaboration partnerships, both national and
international, and is experienced in managing large multi-site projects. BRL has
support from two embedded units specialising in business and enterprise,
together with an incubator and successful track record of spin-outs.

The BRL also has a long track-record of designing and building new and original
robots (from the BERT humanoid in the FP7 CHRIS project, to micro-robotics and
surgical robots). WizUs will directly benefit of this expertise, which will
ensure a feasible and realistic technical deployments of the WizUs robots.
Dedicated technician time is allocated to this end.

The BRL also include a hardware incubator and is co-located with 70 start-ups
and SMEs specialising in robotic hardware and mechatronics (Bristol’s
FutureSpace). This combination of excellent research and vast industry expertise
on one site is unique in the UK, and is will play an instrumental role in
providing opportunities beyond the project towards a strong pathway to impact,
including further engagement with industrial partners and spin-off
opportunities.

\subsection{Other in-kind contributions}

The Bristol science centre will provide in-kind training in science
communication, as well as in-kind access to the centre facilities, for the
duration of the study. The training (10 days in total) would have normally
been billed £3,000 by the science centre.


\section{Risks \& mitigations}




